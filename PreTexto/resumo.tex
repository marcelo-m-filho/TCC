%%%% RESUMO
%%
%% Apresentação concisa dos pontos relevantes de um texto, fornecendo uma visão rápida e clara do conteúdo e das conclusões do
%% trabalho.

\begin{resumoutfpr}%% Ambiente resumoutfpr
O objetivo do projeto é o desenvolvimento e montagem de um aparelho capaz de alterar e aprimorar sinais de som de notebooks e celulares quando conectados a um fone de ouvido. Através de processamento digital de sinais e conversão/amplificação de qualidade, o produto consegue reduzir gargalos no processo de entrega de áudio desde sua versão digital armazenada até o passo final analógico. Sua principal utilidade é na melhoria de fones de ouvido low e mid end, que com o tratamento realizado por este produto poderão obter uma experiência de som melhor sem necessidade de equipamentos de alto custo. Para isso, foi desenvolvido um aparelho também portátil e autosuficiente (i.e. sem necessidade de drivers ou software específico na plataforma na qual está se conectando), para permitir maior flexibilidade em seu uso. O aparelho conta com opções de alterações lineares ao sinal (como volume), além da aplicação de efeitos mais complexos através da composição de filtros, como a equalização.
\end{resumoutfpr}
