%%% CAPÍTULO 1 - INTRODUÇÃO
%%
%% Deve apresentar uma visão global da pesquisa, incluindo: breve histórico, importância e justificativa da escolha do tema,
%% delimitações do assunto, formulação de hipóteses e objetivos da pesquisa e estrutura do trabalho.

%% Título e rótulo de capítulo (rótulos não devem conter caracteres especiais, acentuados ou cedilha)
\chapter{Introdução}\label{cap:introducao}

Um texto curto apresentando o capítulo.

\section{Áudio, gravação e reprodução}\label{sec:audioGravacaoReproducao}
\subsection{Sobre gravação e reprodução de áudio}\label{subsec:gravacaoReproducaoAudio}

A gravação e reprodução de áudio se refere ao processo de ondas sonoras, por meio de canais mecânicos, elétricos, eletrônicos ou digitais. As duas principais vertentes atuais são as de gravação analógica e gravação digital. 

\color{orange}
Existiram gravações de música antes das primeiras gravações de áudio (e.g. caixas de música)

Gravações de música iniciais eram diretamente analógicas

Gravações digitais surgiram e tiveram pushback inicial, porém hoje se tornaram o padrão

\color{black}

% Um texto curto\footnote{Teste de nota de rodapé 1.} apresentando a seção.

\subsection{Subjetividade da percepção do som}\label{subsec:subjetividadeSom}

Aumentar a fidelidade da reprodução de áudio não necessariamente implica em uma maior percepção de qualidade. 

Reproduções imperfeitas, como as provenientes do vinil, são muitas vezes descritos como mais “ricos”, ou mais “quente”, ou “orgânico”. Esses temos qualitativos são utilizados para a descrição de distorções, irregularidades de fase, micro-ajustes de tom e transientes harmônicos que são adicionados no processo de gravação e no de reprodução. 

Embora não sejam individualmente perceptíveis e quantizados ao serem ouvidos, a soma dessas mudanças ao som original levam a uma percepção diferente, e muitas vezes favorável, do som original.

% https://www.londonsinginginstitute.co.uk/why-do-we-like-imperfections-in-music/
% https://www.theparisreview.org/blog/2017/04/21/surface-noise/
% https://www.yoursoundmatters.com/vinyls-imperfections-improve-listening-experience/

A fisiologia humana também tem sua parcela de participação na percepção de sons. Fatores como o tamanho e formato de nossa cabeça, orelhas e canal auditivo, além da densidade das cavidades nasais e orais, transformam o som original e como ele é percebido após chegar em nossas orelhas.  A resposta em frequência de nossos ouvidos varia inclusive entre os dois ouvidos de um mesmo indivíduo.

% https://synrg.csl.illinois.edu/papers/UNIQ_Sigcomm21.pdf
% https://en.wikipedia.org/wiki/Head-related_transfer_function

\subsection{Processamento de sinal para alterações favoráveis}\label{subsec:processamentoSinalAlteracoesFavoraveis}
\subsection{Importância da portabilidade}\label{subsec:importanciaPortabilidade}

\color{orange}
! Consumo de forma distribuída (cloud)

! Possuímos diversos aparelhos diferentes capaz de reprodução de áudio, e os utilizamos em momentos diferentes (em casa, no transporte, no trabalho etc)

! Dispositivos de melhora de áudio feitos sob medida para certos dispositivos (e.g. notebooks com windows) acabam não sendo utilizados 100\% do tempo

\color{black}

\section{Objetivos e Proposta}
\subsection{Processamento digital de sinais}
! Conseguimos simular "problemas" desejáveis (como as distorções do vinil)

! Conseguimos compensar as falhas do dispositivo de reprodução (como uma resposta em frequência não linear)

! Conseguimos personalizar a experiência do usuário de acordo com seus gostos
\subsection{Nicho entre audiófilos que não abrem mão da portabilidade}
! Dispositivos atuais apresentam baixa qualidade

! Houve muita troca de qualidade por comodidade

! Com avanço da tecnologia, podemos ter os dois novamente

! Ter um dispositivo só que possa ser carregado e utilizado em todos os reprodutores de áudio reduz a fricção para utilização 

